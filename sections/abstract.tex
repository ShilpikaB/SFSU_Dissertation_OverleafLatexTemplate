% (This file is included by thesis.tex; you do not latex it by itself.)
\begin{abstract}

% The text of the abstract goes here.  If you need to use a \section
% command you will need to use \section*, \subsection*, etc. so that
% you don't get any numbering.  You probably won't be using any of
% these commands in the abstract anyway.

T-cells are one of the key components of the adaptive immune system. T-cell Receptors (TCR) are a group of protein complexes found on the surface of T-cells. TCRs are responsible for recognizing and binding to certain antigens found on abnormal cells or potentially harmful pathogens. Once the TCRs bind to the pathogens, the T-cells attack these cells and help the body fight infection, cancer, or other diseases. TCR repertoires, which are continually shaped throughout the lifetime of an individual in response to pathogenic exposure, can serve as a fingerprint of an individual’s current immunological profile. The similarity among TCRs sequence directly influences the antigen recognition breadth. Network analysis, which allows interrogation of sequence similarity, thereby adds an important layer of information. Due to the heterogeneous nature of TCR network properties, it is extremely difficult to perform statistical inference or machine learning directly between subjects. In this work, a novel method is proposed to prioritize the network properties that are associated with the outcome of interest, based on features extracted from heterogeneous global/local network properties. Schemes to select the top features associated and to simulate the network properties using the real data are also presented.  Extensive simulation studies and real data analysis were performed to demonstrate the proposed methods. Performance measures including F-1 score, false discovery rate, sensitivity, power, and stability were calculated for each model and are used for model comparison.

\end{abstract}
