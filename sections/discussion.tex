% (This file is included by thesis.tex; you do not latex it by itself.)
\chapter{Discussion}
All analyses and implementations were done using the R programming language with base packages and the subsequent analysis-specific packages. The code is available on \href{https://github.com/ShilpikaB/Prioritizing-Network-Properties-of-T-Cell-Receptors}{\textbf{GitHub}} for reference.
\section{Inferences from Real Data Analysis}\label{sec:realdat_ana}
This thesis focuses on the application of variable selection techniques like Lasso, Group Lasso and Exclusive Lasso using different hyperparameter tuning techniques --- cross-validation and permutation assisted tuning. Using the original sample data, the Group Lasso\_CV model was able to prioritize three of the network properties (feature blocks) --- \lq Membership' (\# of Clusters), \lq Count\_PRE\_INFUSION', and \lq Count\_DOSE\_2'. The Group Plasso model selected only two of the network properties --- \lq Membership' (\# of Clusters) and \lq Count\_PRE\_INFUSION'. The difference in the output of the two models can be attributed to the characteristics of the permutation tuning of having lower false positives (\cite{permassisttune}Yang \textit{et al.}, 2020) in comparison to the cross-validation technique. Therefore, the Group Plasso model appears to be more stringent than Group Lasso\_CV in performing variable selection on the network feature blocks. When using Lasso\_CV and Plasso both models selected the same set of network features as the top performing variables --- the \lq maximum' summary statistics for the TCR network properties \lq Count\_PRE\_INFUSION', \lq diam\_length', \lq eigen\_centrality', \lq centr\_eigen'. Exclusive Lasso (using only cross-validation) model reaffirmed the findings from Lasso\_CV and Plasso by selecting the same set of network features from their corresponding feature blocks.\par

\section{Model Performance Comparison} \label{sec:perf_measure}
From the simulation study on the Group Lasso\_CV and Group Plasso models it is observed that the Group Plasso model has a higher F1 score value. Given that the F1 score aggregates the results from the \lq Sensitivity' and the \lq FDR' values, a higher F1 score is desirable. The Group Plasso model has a stability of $\sim 81\%$ while Group Lasso\_CV has only $\sim 59\%$ stability. This shows that permutation assisted tuning improved the model stability significantly over cross-validation technique.\par
Similarly, when comparing the Lasso\_CV and the Plasso models, the latter has a much lower FDR value and hence a higher F1 score. This aligns with the advantage of permutation assisted tuning having lower false positives than cross-validation technique. Being able to lower the false positives while having the other performance measures almost similar to that of Lasso\_CV, makes the Plasso model a preferable choice for variable selection than Lasso\_CV.\par
Formally verification is required to prove whether the performance of permutation assisted tuning is superior to that of cross-validation for all variable selection problems.\par 

\section{Significant TCR Network Properties and Features} \label{sec:add_features}
Referring to the output from the \autoref{tab:grp_lasso_vs_lasso_sim_study}, the aggregated list of group indexes selected by both Group Lasso\_CV and Group Plasso models are - \lq Membership' (\# of Clusters), \lq Count\_PRE\_INFUSION', and \lq Count\_DOSE\_2'. These feature blocks correspond to the most significant TCR network properties. Similarly, using Lasso\_CV and Plasso models on the original data identify the \lq maximum' summary statistics for \lq Count\_PRE\_INFUSION', \lq diam\_length', \lq eigen\_centrality', \lq centr\_eigen' emerge as the top performing TCR network features.\par